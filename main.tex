%%%%%%%%%%%%%%%%%%%%%%%%%%%%%%%%%%%%%%%%%
% Masters/Doctoral Thesis
%%%%%%%%%%%%%%%%%%%%%%%%%%%%%%%%%%%%%%%%%

%----------------------------------------------------------------------------------------
%	PACKAGES AND OTHER DOCUMENT CONFIGURATIONS
%----------------------------------------------------------------------------------------

\documentclass[11pt, twoside]{Thesis} % The default font size and two-sided printing 
% For a one-sided printing change the flag "twoside" to "oneside" 

\graphicspath{{Pictures/}} % Specifies the directory where pictures are stored

\usepackage[square, numbers, comma, sort&compress]{natbib} % Use the natbib reference package - read up on this to edit the reference style; if you want text (e.g. Smith et al., 2012) for the in-text references (instead of numbers), remove 'numbers' 

%%% BIB STYLE
%\bibliographystyle{apsrev4-1-etal} % With emphasized titles
\bibliographystyle{apsrev4-1}


\hypersetup{colorlinks,citecolor=green,urlcolor=blue}


%%% FONT PACKAGES
% - fourier
% - mathpazo ( palatino for Computer Modern on math )
\usepackage{mathpazo}


\newcommand*{\defaultlinespacing}{\setstretch{1.5}}
\newcommand*{\listlinespacing}{\setstretch{1.7}}

\usepackage{lipsum} % For dummy text


%----------------------------------------------------------------------------------------
%	DOCUMENT VARIABLES
%----------------------------------------------------------------------------------------

\thesistitle{Thesis Title} % Your thesis title \ttitle
\thesistype{Masters Thesis} % Your thesis type Doctoral Thesis or Masters Thesis \ttype
\supervisor[mailto:name@host.com]{John \textsc{Doe}} % Your supervisor's name \supname \supnamenolink
\cosupervisor[mailto:name@host.com]{John \textsc{Doe}} % Your supervisor's name \cosupname \cosupnamenolinkTo
% To hide the Co-Supervisor field, just comment out \cosupervisor
\degree{Master of Science} % Your degree name \degreename 
\authors[mailto:name@host.com]{John \textsc{Doe}} % Your name \authornames \authornamesnolink
\addresses{} % Your address \addressname
\subject{} % Your subject area \subjectname
\keywords{} % Keywords for your thesis \keywordnames 
\university[http://www.univesityurl.com]{University} % Your university's name \univname \univnamenolink
\UNIVERSITY[http://www.univesityurl.com]{UNIVERSITY} % Your university's name in capitals \UNIVNAME \UNIVNAMEnolink             
\department[http://www.departmenturl.com]{Departament} % Your department's name \deptname \deptnamenolink
\DEPARTMENT[http://www.departmenturl.com]{DEPARTAMENT} % Your department's name in capitals \DEPTNAME \DEPTNAMEnolink
% To hide the Department field, just comment out \DEPARTMENT and \department        
\group[http://www.groupurl.com]{Research Group} % Your research group's name \groupname \groupnamenolink           
\GROUP[http://www.groupurl.com]{RESEARCH GROUP} % Your research group's name in capitals \GROUPNAME \GROUPNAMEnolink
% To hide the Group field, just comment out \GROUP and \group   
\faculty[http://www.facultyurl.com]{Faculty} % Your faculty's name \facname \facnamenolink
\FACULTY[http://www.facultyurl.com]{FACULTY} % Your faculty's name in capitals \FACNAME \FACNAMEnolink
% To hide the Faculty field, just comment out \FACULTY and \Faculty     
% NOTE: To remove links write \cmd{name} instead of \cmd[link]{name}
% NOTE: For aesthetics, at least one of these fields must be set: faculty, group or department



%----------------------------------------------------------------------------------------
%	DOCUMENT
%----------------------------------------------------------------------------------------

\begin{document}

\frontmatter % Use roman page numbering style (i, ii, iii, iv...) for the pre-content pages

\defaultlinespacing % Line spacing of 1.5


%----------------------------------------------------------------------------------------
%	TITLE PAGE
%----------------------------------------------------------------------------------------

\maketitle

%----------------------------------------------------------------------------------------
%	QUOTATION PAGE
%----------------------------------------------------------------------------------------

\quotepage{Palavra do Dominum}
{
	Naquele tempo, depois da lição Parental, Dominum falou aos seus projectos de apóstolos das criaturas que vaguavam: ``\emph{As tipas andam de relationship em relationship. Aquilo é falta de Pai}''. Um dos apóstolos, de seu nome Pedro, perguntou-lhe ``\emph{Pai? Como assim, Pai?}'', ao que Dominum lhe respondeu: ``\emph{Pai? Pai há só um!}''. E, com isto, Dominum revelou-se aos seus projectos de apóstolos como Pai de todos, e estes prosseguiram a sua vida felizes com o que lhes foi dito.
}


%----------------------------------------------------------------------------------------
%	ACKNOWLEDGEMENTS
%----------------------------------------------------------------------------------------

\acknowledgements{\addtocontents{toc}{\vspace{1em}} % Add a gap in the Contents, for aesthetics
	
	The acknowledgements and the people to thank go here, don't forget to include your project advisors\ldots

}


%----------------------------------------------------------------------------------------
%	ABSTRACT PAGE
%----------------------------------------------------------------------------------------

\addtotoc{Abstract} % Add the "Abstract" page entry to the Contents

\abstract
{
	\addtocontents{toc}{\vspace{1em}} % Add a gap in the Contents, for aesthetics
	
	The Thesis Abstract is written here (and usually kept to just this page). The page is kept centered vertically so can expand into the blank space above the title too\ldots
	
	\lipsum[2]
}

\cleardoublepage


%----------------------------------------------------------------------------------------
%	ABSTRACT PAGE (PORTUGUESE)
%----------------------------------------------------------------------------------------

\addtotoc{Resumo} % Add the "Resumo" page entry to the Contents

\abstract[title=Resumo,degree={Mestre de Ciência},connector=por]
{
	\addtocontents{toc}{\vspace{1em}} % Add a gap in the Contents, for aesthetics
	
	Tradução em português do ``Abstract'' escrito em inglês mais a cima. A página é centrada vertical  e horizontalmente, podendo espandir para o espaço superior da página em branco \ldots
	
	\lipsum[2-3]
}

\cleardoublepage


%----------------------------------------------------------------------------------------
%	LIST OF CONTENTS/FIGURES/TABLES PAGES
%----------------------------------------------------------------------------------------

\tableofcontents % Write out the Table of Contents

\listoffigures % Write out the List of Figures

\listoftables % Write out the List of Tables

\addtocontents{toc}{\vspace{1em}}


%----------------------------------------------------------------------------------------
%	ABBREVIATIONS
%----------------------------------------------------------------------------------------

\listlinespacing % This makes the following tables easier to read

\listofabbreviations{r@{\hskip 0.4in}l} % Include a list of Abbreviations (a table of two columns)
{
	\textbf{GR}		& \textbf{G}eneral \textbf{R}elativity \\
	\textbf{BH}		& \textbf{B}lack \textbf{H}ole \\
	
	\hspace*{0.9in} & \hspace*{5in} % Some structure to the table
}

\defaultlinespacing % Return the line spacing back to default


%----------------------------------------------------------------------------------------
%	PHYSICAL CONSTANTS/OTHER DEFINITIONS
%----------------------------------------------------------------------------------------

\listlinespacing % This makes the following tables easier to read

\listofconstants{r@{\hskip 0.4in}lcl} % a four column table
{
	Speed of Light & $c$ & $=$ & $2.997\ 924\ 58\times10^{8}\ \mbox{ms}^{-\mbox{s}}$ (exact)\\
	% Constant Name & Symbol & $=$ & Constant Value (with units) \\
}

\defaultlinespacing % Return the line spacing back to default


%----------------------------------------------------------------------------------------
%	SYMBOLS
%----------------------------------------------------------------------------------------

\listlinespacing % This makes the following tables easier to read

\listofsymbols{l@{\hskip 0.4in}l@{\hskip 0.4in}l} % a three column table
{
	$a$ & distance & m \\
	$P$ & power & W (Js$^{-1}$) \\
	% Symbol & Name & Unit \\

	& & \\ % Gap to separate the Roman symbols from the Greek

	$\omega$ & angular frequency & rads$^{-1}$ \\

	% Symbol & Name & Unit \\
}

\defaultlinespacing % Return the line spacing back to default


%----------------------------------------------------------------------------------------
%	NOTATION
%----------------------------------------------------------------------------------------
\newcommand\notationname{Notation and Conventions}

\addtotoc{\notationname}
\chapter*{\notationname}
\fancyhead[LO]{\textsc{\notationname}}

\section*{Units}

Text random and a new citation

\section*{Tensors and relativity related}

\lipsum[1-10] % Random text

\cleardoublepage


%----------------------------------------------------------------------------------------
%	DEDICATORY
%----------------------------------------------------------------------------------------

\dedicatory{For/Dedicated to/To my\ldots} % Dedication text

\addtocontents{toc}{\vspace{2em}} % Add a gap in the Contents, for aesthetics

\cleardoublepage


%----------------------------------------------------------------------------------------
%	THESIS CONTENT - CHAPTERS
%----------------------------------------------------------------------------------------

\addtocontents{toc}{\vspace{1em}}

\mainmatter % Begin numeric (1,2,3...) page numbering

\pagestyle{fancy} % Return to usual headers
\renewcommand{\chaptermark}[1]{\markboth{\thechapter. \textsc{#1}}{}} % Change the format of the \leftmark to appear "number. chaper name"
\fancyhead[LO]{\leftmark} % Make only the left odd pages header to appear the chapter name
%This will be the default header after changing the flag "twoside" to "oneside" 


\input{Chapters/Chapter1}
%% Chapter 1

\chapter{Chapter 2} % Main chapter title
\label{Chapter2} % For referencing the chapter elsewhere, use \ref{Chapter1} 


\lipsum[1-20]


\cleardoublepage
 
%\input{Chapters/Chapter3}

%----------------------------------------------------------------------------------------
%	THESIS CONTENT - APPENDICES
%----------------------------------------------------------------------------------------

\addtocontents{toc}{\vspace{2em}} % Add a gap in the Contents, for aesthetics

\appendix % Cue to tell LaTeX that the following 'chapters' are Appendices

% Appendix A

\chapter{Appendix A} % Main appendix title

\label{AppendixA} % For referencing this appendix elsewhere, use \ref{AppendixA}

\lipsum[30]


\cleardoublepage
%% Appendix A

\chapter{Appendix B} % Main appendix title

\label{AppendixB} % For referencing this appendix elsewhere, use \ref{AppendixA}

\lipsum[31]


\cleardoublepage

\addtocontents{toc}{\vspace{2em}} % Add a gap in the Contents, for aesthetics

\backmatter


%----------------------------------------------------------------------------------------
%	BIBLIOGRAPHY
%----------------------------------------------------------------------------------------

\label{Bibliography}

\fancyhead[LO]{\textsc{Bibliography}}

\nocite{*} % CITE ALL REFERENCES
\bibliography{Bibliography.bib} % The references (bibliography) information are stored in the file named "Bibliography.bib"

\end{document}
