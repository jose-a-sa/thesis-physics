%%%%%%%%%%%%%%%%%%%%%%%%%%%%%%%%%%%%%%%%%
% Masters/Doctoral Thesis
%%%%%%%%%%%%%%%%%%%%%%%%%%%%%%%%%%%%%%%%%

%----------------------------------------------------------------------------------------
%	PACKAGES AND OTHER DOCUMENT CONFIGURATIONS
%----------------------------------------------------------------------------------------

\documentclass[11pt, twoside]{Thesis} % The default font size and two-sided printing 
% For a one-sided printing change the flag "twoside" to "oneside" 

\usepackage{pdfpages}

\graphicspath{{Pictures/}} % Specifies the directory where pictures are stored

\usepackage[square, numbers, comma, sort&compress]{natbib} % Use the natbib reference package - read up on this to edit the reference style; if you want text (e.g. Smith et al., 2012) for the in-text references (instead of numbers), remove 'numbers' 

%%% BIB STYLE
%\bibliographystyle{apsrev4-1-etal} % With emphasized titles
\bibliographystyle{apsrev4-1}

\hypersetup{colorlinks,citecolor=green,urlcolor=blue,linkcolor=red}

%%% FONT PACKAGES
% - fourier
% - mathpazo ( palatino for Computer Modern on math )
\usepackage{mathpazo}
\setstretch{1.5}


\usepackage{lipsum} % For dummy text


%----------------------------------------------------------------------------------------
%	DOCUMENT VARIABLES
%----------------------------------------------------------------------------------------

\thesistitle{Thesis Title} % Your thesis title \ttitle
\thesistype{Masters Thesis} % Your thesis type Doctoral Thesis or Masters Thesis \ttype
\supervisor[mailto:name@host.com]{John \textsc{Doe}} % Your supervisor's name \supname \supnamenolink
\cosupervisor[mailto:name@host.com]{John \textsc{Doe}} % Your supervisor's name \cosupname \cosupnamenolinkTo
% To hide the Co-Supervisor field, just comment out \cosupervisor
\degree{Master of Science} % Your degree name \degreename 
\authors[mailto:name@host.com]{John \textsc{Doe}} % Your name \authornames \authornamesnolink
\addresses{} % Your address \addressname
\subject{} % Your subject area \subjectname
\keywords{} % Keywords for your thesis \keywordnames 
\university[http://www.univesityurl.com]{University} % Your university's name \univname \univnamenolink
\UNIVERSITY[http://www.univesityurl.com]{UNIVERSITY} % Your university's name in capitals \UNIVNAME \UNIVNAMEnolink             
\department[http://www.departmenturl.com]{Departament} % Your department's name \deptname \deptnamenolink
\DEPARTMENT[http://www.departmenturl.com]{DEPARTAMENT} % Your department's name in capitals \DEPTNAME \DEPTNAMEnolink
% To hide the Department field, just comment out \DEPARTMENT and \department        
\group[http://www.groupurl.com]{Research Group} % Your research group's name \groupname \groupnamenolink           
\GROUP[http://www.groupurl.com]{RESEARCH GROUP} % Your research group's name in capitals \GROUPNAME \GROUPNAMEnolink
% To hide the Group field, just comment out \GROUP and \group   
\faculty[http://www.facultyurl.com]{Faculty} % Your faculty's name \facname \facnamenolink
\FACULTY[http://www.facultyurl.com]{FACULTY} % Your faculty's name in capitals \FACNAME \FACNAMEnolink
% To hide the Faculty field, just comment out \FACULTY and \Faculty     
% NOTE: To remove links write \cmd{name} instead of \cmd[link]{name}
% NOTE: For aesthetics, at least one of these fields must be set: faculty, group or department


%----------------------------------------------------------------------------------------
%	NOTATION VARIABLES
%----------------------------------------------------------------------------------------

\newcommand{\dd}{\mathrm{d}}


%----------------------------------------------------------------------------------------
%	DOCUMENT
%----------------------------------------------------------------------------------------

\begin{document}

\frontmatter % Use roman page numbering style (i, ii, iii, iv...) for the pre-content pages


%----------------------------------------------------------------------------------------
%	TITLE PAGE
%----------------------------------------------------------------------------------------

\maketitle

%----------------------------------------------------------------------------------------
%	QUOTATION PAGE
%----------------------------------------------------------------------------------------

\quotepage{Palavra do Dominum}
{
	Naquele tempo, depois da lição Parental, Dominum falou aos seus projectos de apóstolos das criaturas que vaguavam: ``\emph{As tipas andam de relationship em relationship. Aquilo é falta de Pai}''. Um dos apóstolos, de seu nome Pedro, perguntou-lhe ``\emph{Pai? Como assim, Pai?}'', ao que Dominum lhe respondeu: ``\emph{Pai? Pai há só um!}''. E, com isto, Dominum revelou-se aos seus projectos de apóstolos como Pai de todos, e estes prosseguiram a sua vida felizes com o que lhes foi dito.
}


%----------------------------------------------------------------------------------------
%	ACKNOWLEDGEMENTS
%----------------------------------------------------------------------------------------

\begin{acknowledgements}
	\lipsum[1]
\end{acknowledgements}

\addvspacetoc{0.3cm} % Add a gap in the Contents, for aesthetics


%----------------------------------------------------------------------------------------
%	ABSTRACT PAGE
%----------------------------------------------------------------------------------------

\begin{abstract}

	The Thesis Abstract is written here (and usually kept to just this page). The page is kept centered vertically so can expand into the blank space above the title too\ldots
	
\end{abstract}


%----------------------------------------------------------------------------------------
%	ABSTRACT PAGE (PORTUGUESE)
%----------------------------------------------------------------------------------------

\begin{abstract}[
	title={Resumo},
	degree={Mestre de Ciência},
	nameconnector={por}]

	Tradução em português do ``Abstract'' escrito em inglês mais a cima. A página é centrada vertical  e horizontalmente, podendo espandir para o espaço superior da página em branco \ldots
	
\end{abstract}


%----------------------------------------------------------------------------------------
%	LIST OF CONTENTS/FIGURES/TABLES
%----------------------------------------------------------------------------------------

\addvspacetoc{0.3cm}

\tableofcontents % Write out the Table of Contents

\listoffigures % Write out the List of Figures

\listoftables % Write out the List of Tables

\addvspacetoc{0.3cm}

%----------------------------------------------------------------------------------------
%	PHYSICAL CONSTANTS/OTHER DEFINITIONS
%----------------------------------------------------------------------------------------

\begin{listofcontants}
	\const{My little ponny test of magical rainbow}{$mn/mp$}{$2.997\ 924\ 58\times10^{8}\ \mbox{ms}^{-\mbox{s}}$}
	\const{Vaccuum permeability test of magical rainbow for a specific case of condensed matter physics}{$\epsilon_0$}{$2.997\ 924\ 58\times10^{8}\ \mbox{ms}^{-\mbox{s}}$}
	\const{Speed of Light test of magical rainbow}{$c$}{$2.997\ 924\ 58\times10^{8}\ \mbox{ms}^{-\mbox{s}}$}
\end{listofcontants}


%----------------------------------------------------------------------------------------
%	SYMBOLS
%----------------------------------------------------------------------------------------

\begin{listofsymbols}
	\symb{$F_{\mu\nu}$}{Maxwell tensor}{F}
	\symb{$a$}{distance}{m}
	\\
	\symb{$\omega$}{angular frequency}{rads$^{-1}$}
\end{listofsymbols}


%----------------------------------------------------------------------------------------
%	NOTATION
%----------------------------------------------------------------------------------------

% \newcommand\notationname{Notation and Conventions}
% \addtotoc{\notationname}
% \fancyhead[LO]{\textsc{\notationname}}

% \input{Notation}



%----------------------------------------------------------------------------------------
%	ABBREVIATIONS
%----------------------------------------------------------------------------------------

\begin{glossary}
	\abbrev{QM}{Quantum Mechanics}
\end{glossary}


%----------------------------------------------------------------------------------------
%	DEDICATORY
%----------------------------------------------------------------------------------------

\begin{dedicatory}
	For/Dedicated to/To my\ldots
\end{dedicatory}


%----------------------------------------------------------------------------------------
%	THESIS CONTENT - CHAPTERS
%----------------------------------------------------------------------------------------

\addvspacetoc{0.3cm}

\mainmatter % Begin numeric (1,2,3...) page numbering

\pagestyle{fancy}
\renewcommand{\chaptermark}[1]{\markboth{\thechapter. \textsc{#1}}{}}
\fancyhead[LO]{\leftmark}

\input{Chapters/Chapter1}
%% Chapter 1

\chapter{Chapter 2} % Main chapter title
\label{Chapter2} % For referencing the chapter elsewhere, use \ref{Chapter1} 


\lipsum[1-20]


\cleardoublepage
 
%\input{Chapters/Chapter3}
%\input{Chapters/Chapter4} 
%\input{Chapters/Chapter5} 



%----------------------------------------------------------------------------------------
%	THESIS CONTENT - APPENDICES
%----------------------------------------------------------------------------------------

\appendix % Cue to tell LaTeX that the following 'chapters' are Appendices

% Appendix A

\chapter{Appendix A} % Main appendix title

\label{AppendixA} % For referencing this appendix elsewhere, use \ref{AppendixA}

\lipsum[30]


\cleardoublepage
%% Appendix A

\chapter{Appendix B} % Main appendix title

\label{AppendixB} % For referencing this appendix elsewhere, use \ref{AppendixA}

\lipsum[31]


\cleardoublepage
%\input{Appendices/AppendixC}

\backmatter

%----------------------------------------------------------------------------------------
%	BIBLIOGRAPHY
%----------------------------------------------------------------------------------------

\addvspacetoc{0.5cm}
\addtotoc{Bibliography}

\fancyhead[LO]{\textsc{Bibliography}}

\nocite{*}
\bibliography{../Bibliography.bib} % The references are stored in the file "Bibliography.bib"


\end{document}