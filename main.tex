%%%%%%%%%%%%%%%%%%%%%%%%%%%%%%%%%%%%%%%%%
% Masters/Doctoral Thesis
%%%%%%%%%%%%%%%%%%%%%%%%%%%%%%%%%%%%%%%%%

%-----------------------------------------------------------------------------------
%	PACKAGES AND OTHER DOCUMENT CONFIGURATIONS
%-----------------------------------------------------------------------------------

\documentclass[10pt, twoside]{Thesis} % The default font size and two-sided printing 
% For a one-sided printing change the flag "twoside" to "oneside" 

\usepackage{pdfpages}

\graphicspath{{Pictures/}} % Specifies the directory where pictures are stored

\usepackage[square, numbers, comma, sort&compress]{natbib} % Use the natbib reference package - read up on this to edit the reference style; if you want text (e.g. Smith et al., 2012) for the in-text references (instead of numbers), remove 'numbers' 

%%% BIB STYLE
%\bibliographystyle{apsrev4-1-etal} % With emphasized titles
\bibliographystyle{apsrev4-1}


%%% FONT PACKAGES
% - fourier
% - mathpazo ( palatino for Computer Modern on math )
\usepackage{mathpazo}
\setstretch{1.4}


\usepackage{lipsum} % For dummy text


%-----------------------------------------------------------------------------------
%	DOCUMENT VARIABLES
%-----------------------------------------------------------------------------------

\thesistitle[Small title]{Very long long long long long long long long long long long long long long long a and complete title}
\author[mailto:name@host.com]{John \textsc{Doe}} % Your name
% \address{} % Your address
% \subject{} % Your subject area
% \keywords{} % Keywords for your thesis 
\supervisor[mailto:name@host.com]{John \textsc{Doe}} % Your supervisor's name
\cosupervisor[mailto:name@host.com]{John \textsc{Doe}} % Your co-supervisor's name
% To hide the co-supervisor field, just comment out \cosupervisor
\degree[Master's]{MSc}{Master of Science} % Your degree name
\university[http://www.univesityurl.com]{University} % Your university's name
\department[http://www.departmenturl.com]{Departament} % Your department's name 
% To hide the Department field, just comment out \department        
\group[http://www.groupurl.com]{Research Group} % Your research group's name
% To hide the Group field, just comment out \group   
\faculty[http://www.facultyurl.com]{Faculty} % Your faculty's name \facname
% To hide the Faculty field, just comment out faculty     
% To remove links write \cmd{name} instead of \cmd[link]{name}
% For aesthetics, at least one of these fields must be set: faculty, group or department


%-----------------------------------------------------------------------------------
%	NOTATION VARIABLES
%-----------------------------------------------------------------------------------

\newcommand{\dd}{\mathrm{d}}
\newcommand{\fractd}[2][]{\frac{\dd #1}{#2}}
\newcommand{\fracpd}[2][]{\frac{\partial #1}{\partial #2}}


%-----------------------------------------------------------------------------------
%	DOCUMENT
%-----------------------------------------------------------------------------------

\begin{document}

\frontmatter % Use roman page numbering style (i, ii, iii, iv...) for the pre-content pages

\maketitle


%-----------------------------------------------------------------------------------
%	QUOTATION PAGE
%-----------------------------------------------------------------------------------

\quotepage{Palavra do Dominum}{%
	Naquele tempo, depois da lição Parental, Dominum falou aos seus projectos de apóstolos das criaturas que vaguavam: ``\emph{As tipas andam de relationship em relationship. Aquilo é falta de Pai}''. Um dos apóstolos, de seu nome Pedro, perguntou-lhe ``\emph{Pai? Como assim, Pai?}'', ao que Dominum lhe respondeu: ``\emph{Pai? Pai há só um!}''. E, com isto, Dominum revelou-se aos seus projectos de apóstolos como Pai de todos, e estes prosseguiram a sua vida felizes com o que lhes foi dito.}


%-----------------------------------------------------------------------------------
%	ACKNOWLEDGEMENTS
%-----------------------------------------------------------------------------------

\begin{acknowledgements}
	\lipsum[1]
\end{acknowledgements}

\addvspacetoc{3mm} % Add a gap in the Contents, for aesthetics


%-----------------------------------------------------------------------------------
%	ABSTRACT PAGE
%-----------------------------------------------------------------------------------

\begin{abstract}
	\lipsum[10-20]
\end{abstract}


%-----------------------------------------------------------------------------------
%	ABSTRACT PAGE (PORTUGUESE)
%-----------------------------------------------------------------------------------

\begin{abstract}[title={Resumo},degree={Mestre de Ciência},nameconnector={por}]
	Tradução em português do ``Abstract'' escrito em inglês mais a cima. A página é centrada vertical  e horizontalmente, podendo espandir para o espaço superior da página em branco \ldots
\end{abstract}


%-----------------------------------------------------------------------------------
%	LIST OF CONTENTS/FIGURES/TABLES
%-----------------------------------------------------------------------------------

\addvspacetoc{3mm} % Add a gap in the Contents, for aesthetics

\tableofcontents % Write out the Table of Contents

\listoffigures % Write out the List of Figures

\listoftables % Write out the List of Tables

\addvspacetoc{3mm} % Add a gap in the Contents, for aesthetics

%-----------------------------------------------------------------------------------
%	PHYSICAL CONSTANTS/OTHER DEFINITIONS
%-----------------------------------------------------------------------------------

\begin{listofcontants}
	\const{My little ponny test of magical rainbow}{$m_n/m_p$}{$2.997\ 924\ 58\times10^{8}\ \text{m\,s}^{-1}$}
	\const{Vaccuum permeability test of magical rainbow for a specific case of condensed matter physics}{$\epsilon_0$}{$2.997\ 924\ 58\times10^{8}\ \text{m\,s}^{-1}$}
	\const{Speed of Light test of magical rainbow}{$c$}{$2.997\ 924\ 58\times10^{8}\ \text{m\,s}^{-1}$}
\end{listofcontants}


%-----------------------------------------------------------------------------------
%	SYMBOLS
%-----------------------------------------------------------------------------------

\begin{listofsymbols}
	\symb{$F_{\mu\nu}$}{Maxwell tensor}{F}
	\symb{$a$}{distance}{m}
	\\
	\symb{$\omega$}{angular frequency}{rads$^{-1}$}
\end{listofsymbols}


%-----------------------------------------------------------------------------------
%	NOTATION
%-----------------------------------------------------------------------------------

% \newcommand\notationname{Notation and Conventions}
% \addtotoc{\notationname}
% \fancyhead[LO]{\textsc{\notationname}}

% \input{Notation}



%-----------------------------------------------------------------------------------
%	ABBREVIATIONS
%-----------------------------------------------------------------------------------

\begin{glossary}
	\abbrev{QM}{\b Quantum \b Mechanics}
	\abbrev{LCAO}{\b Linear \b combination of \b atomic \b orbitals}
	\abbrev{GR}{\b General \b Relativity}
\end{glossary}


%-----------------------------------------------------------------------------------
%	DEDICATORY
%-----------------------------------------------------------------------------------

\begin{dedicatory}
	For/Dedicated to/To my\ldots
\end{dedicatory}


%-----------------------------------------------------------------------------------
%	THESIS CONTENT - CHAPTERS
%-----------------------------------------------------------------------------------

\addvspacetoc{3mm} % Add a gap in the Contents, for aesthetics

\mainmatter % Begin numeric (1,2,3...) page numbering

% \pagestyle{fancy}
% \renewcommand{\chaptermark}[1]{\markboth{\thechapter. \textsc{#1}}{}}
% \fancyhead[LO]{\leftmark}

\input{Chapters/Chapter1}
% Chapter 1

\chapter{Chapter 2} % Main chapter title
\label{Chapter2} % For referencing the chapter elsewhere, use \ref{Chapter1} 


\lipsum[1-20]


\cleardoublepage
 
%\input{Chapters/Chapter3}
%\input{Chapters/Chapter4} 
%\input{Chapters/Chapter5} 



%-----------------------------------------------------------------------------------
%	THESIS CONTENT - APPENDICES
%-----------------------------------------------------------------------------------

\appendix % Cue to tell LaTeX that the following 'chapters' are Appendices

% Appendix A

\chapter{Appendix A} % Main appendix title

\label{AppendixA} % For referencing this appendix elsewhere, use \ref{AppendixA}

\lipsum[30]


\cleardoublepage
%% Appendix A

\chapter{Appendix B} % Main appendix title

\label{AppendixB} % For referencing this appendix elsewhere, use \ref{AppendixA}

\lipsum[31]


\cleardoublepage
%\input{Appendices/AppendixC}

\backmatter

%-----------------------------------------------------------------------------------
%	BIBLIOGRAPHY
%-----------------------------------------------------------------------------------

\addvspacetoc{5mm}
\addtotoc{Bibliography}

\fancyhead[LO]{\textsc{Bibliography}}

\nocite{*}
\bibliography{Bibliography.bib} % The references are stored in the file "Bibliography.bib"


\end{document}